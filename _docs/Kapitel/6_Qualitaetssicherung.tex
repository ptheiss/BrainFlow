Da der Auszubildende das einzige Teammitglied am Projekt ist, war ein Code Review nur sporadisch möglich. Stattdessen hat man sich entschieden bei der Aufsetzung
des Quasar-Projekts, die angebotenen Code Formatter und Linter mitzuinstallieren (Prettier + ESLint). Dies gewährleistet, dass der geschriebene Code sich beim
Speichern automatisch gewissen Style Richtlinien anpasst. Der Development Server gibt zur selben Zeit gefundene Fehler im Code in der Konsole aus, was die
Frontendentwicklung sehr unanfällig gegenüber Fehlern gestaltet hat.

\bigskip\noindent
Da ein sehr großer Teil der Komplexität des Projekts von der Implementierung der API zusammenhängt, wurde grundsätzlich entschieden einen Unit Test
auf Seiten des Backends zu schreiben, der überprüft, ob die API-Anbindung erfolgreich war, indem er CRUD-Operationen an die API schickt.

\subsection*{brainFlowTestCase(TestCase)}
Beim Aufsetzen einer Django Applikation wird eine tests.py erstellt.
In dieser tests.py können wir unsere Tests als Klasse implementieren. Mit \flqq python manage.py test\frqq werden alle Funktionen durchgeführt, die mit \flqq test \frqq beginnen.
Da Django von unabhängigen Unit Tests ausgeht, ist die Reihenfolge der Ausführung dieser Tests zufällig gewählt. Da unser Test jedoch von der Reihenfolge der Operationen
abhängig ist, mussten sie in denselben Test gepackt werden.

\bigskip\noindent
Unser Test durchläuft folgende Schritte:
    \begin{itemize}
        \item Es wird eine Notiz in der Datenbank erstellt.
        \item Diese Notiz wird aus der Datenbank gelesen.
        \item Dann wird die Notiz geändert.
        \item Schließlich wird die Notiz gelöscht.
    \end{itemize}

\bigskip\noindent
Django REST Framework stellt ein APIRequestFactory Objekt zur Verfügung, dass das Erstellen von Requests an die API-URLs vereinfacht.
Mit der APIRequestFactory wird auch das NoteViewSet von api.views importiert, da wir anhand dieser Tabelle testen.

\bigskip\noindent
Für das Testen legt Django eine temporäre Testdatenbank an, damit bestehende Daten nicht verfälscht oder gelöscht werden. Alternativ kann man eine
sogenannte \flqq Fixture\frqq verwenden. Fixtures sind prinzipiell Datenbankexports, mit denen man über das manage.py script Daten exportieren oder
importieren kann. Wenn Fixture-Dateien in einem Django Test definiert sind, dann werden diese beim Ausführen des Tests in die Test-Datenbank geladen.