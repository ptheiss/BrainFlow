\section{Ist-Zustand}
        Der Betrieb des Auszubildenden verwendet bereits eine selbst gehostete Dokumentationssoftware namens BookStack.
        Für Inhalte die sich selten ändern, wie Tutorials oder Datenstrukturen, haben die Mitarbeiter mit BookStack kein Problem,
        aber für Kleininformationen die sich oft ändern existiert ein hoher Pflege- und Organisierbedarf um die Lesbarkeit der Dokumentation zu schonen.

        \bigskip\noindent
        Das führte dazu, dass jeder Mitarbeiter verschiedenste Softwareprodukte und Medien verwendet um Informationen aufzuzeichnen, die dann nur lokal
        am PC existieren und nicht verfügbar sind, falls ein Mitarbeiter krank oder im Urlaub ist und falls über Messenger wie Microsoft Teams mitgeteilt, nur 
        schwierig wieder auffindbar sind.

\section{Soll-Analyse}
        Mit BrainFlow soll der Auszubildende ein Pendant zur BookStack-Dokumentation entwickeln, wo Kleininformationen schneller und einfacher aufgezeichnet und
        geteilt werden können, mit der Aussicht auf Erfahrungsgewinnung mit anderen Softwareframeworks. Verschiedene Sachverhalten sollen mit Tags auseinandergehalten werden können.