\section{Projektorganisation}\label{sec:projektorganisation}
    Für das Projekt verwendet der Auszubildende die agile Projektmethode Kanban. Kanban wurde gegenüber Scrum bevorzugt, da nach Ansicht des Auszubildenden
    für die gegebene Zeiteinteilung als einziger Entwickler wenig Sinn darin besteht sich nach Sprints zu orientieren und meherere Rollen zu füllen. Die
    Flexibilität von Kanban was das Updaten des Kanban-Boards angeht (bei Scrum sollte das Scrum-Board grundsätzlich nur beim Sprint Planning angefasst werden)
    spricht auch dafür, dass für das Projekt Kanban verwendet wurde.

    \bigskip\noindent
    Für das Tracking von Tasks wurde auf dem GitHub Repository des Projekts ein Kanban-Board angelegt.

\section{Ressourcenplanung}\label{sec:resourcenplanung}
    Für die Umsetzung des Projekts standen folgende Resourcen zur Verfügung:

    \begin{itemize}
        \item Personal: 1 Anwendungsentwickler
        \item Hardware: 1 Desktop-PC, 1 Server
        \item Software: 
            \begin{itemize}
                \item Betriebssystem: Windows
                \item Visual Studio Code inkl. Git
                \item Microsoft Teams
                \item Webbrowser
            \end{itemize}
        \item Zeit: 80 Stunden (Details in den Anhängen unter \flqq Zeitplanung\frqq)
    \end{itemize}

    \section{Entwicklungsprozess}\label{sec:entwicklungsprozess}
        Für die Frontendentwicklung wird das Framework \flqq Quasar\frqq verwendet. Das Quasar-CLI Tool erlaubt die Aufsetzung eines lokalen Development-Servers,
        durch den die Mehrheit der Entwicklung und Tests getätigt wurde.

        \bigskip\noindent
        Für die Backendentwicklung wird das Framework \flqq Django\frqq verwendet.
        Zur Anbindung der API wird das Plugin \flqq Django Rest Framework\frqq benutzt.
        Mit dem Python-Paket \flqq venv\frqq wird zudem eine virtuelle Umgebung aufgestellt, welche installierte Plugins zu dem Projekt-Ordner isoliert.
        Zur Versionierung wird ein Repository auf \flqq GitHub\frqq angelegt.