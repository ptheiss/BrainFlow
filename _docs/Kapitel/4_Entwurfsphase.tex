\section {Anforderungen und Entwicklungsablauf}
Folgende Anforderungen wurden an der Applikation festgelegt:
    \begin{itemize}
        \item Es soll eine Login-Funktion existieren.
        \item Eingeloggte User sollen Notizen erstellen können.
        \item Notizen und Sachverhalte sollen mit einem Tag-System auseinander gehalten werden.
        \item Es soll einen Button geben, der die Inhalte einer Notiz in die Zwischenablage kopieren kann.
    \end{itemize}

\bigskip\noindent
Früh in der Entwicklung wurde noch der Punkt hinzugenommen, dass User Applikationseinstellungen setzen können. Es war ursprünglich angedacht, dass die Tags
eine Baumstruktur haben, da das in der Datenbank einfach modelliert werden kann, allerdings hatte in der Umsetzung das \flqq Hochklettern\frqq des Baumes in der Darstellung
Probleme bereitet. Als Ersatz wurden Arbeitsgruppen in die Spezifikation dazugenommen, denen sowohl Tags als auch User angehören können.

\bigskip\noindent
Zur Visualisierung wurden ein \hyperref[usecase]{Use Case Diagramm} und ein \hyperref[mockup]{Mockup} erstellt.

\bigskip\noindent
Die Zielplattform von BrainFlow sind Desktop-Computer und Notebooks. Zur Implementierung der Full-Stack Applikation hat sich der Auszubildende entschieden
das Frontendframework Quasar in Tandem mit dem Backendframework Django zu verwenden. Ursprünglich wurde für das Frontend nur Vue.js ausgewählt, da der Auszubildende hiermit
Vorerfahrungen hatte, allerdings stellt Vue von sich aus keine UI-Komponenten zur Verfügung.

\bigskip\noindent
Da die Entwicklung eigener Komponenten einen sehr hohen Zeitaufwand mit sich bringt, musste man sich kurz nach Projektangang doch für eine Erweiterung entscheiden.
Hier wurde erwägt zwischen Vuetify, eine Bibliothek für Vue.js, oder ein anderes Framework zu benutzen. Am Ende konnte sich Quasar, welches ebenso auf Vue.js basiert,
durchsetzen aufgrund der \flqq What You See Is What You Get\frqq-Editor Komponente, welche als Basis für die Notiz-Funktion dient.

\subsection{Frontend}
Quasar basiert auf Vue.js, welches sich durch die Nutzung von Single File Components auszeichnet. Diese stellen sich zusammen aus einer Template Sektion (Vue + HTML),
Script Sektion (JavaScript) und Style Sektion (CSS). SFCs werden für alle Web Design Aspekte von Vue.js verwendet (UI-Element, Toolbars, ganze Webpages) und können in
den Template-Bereich anderer SFCs importiert werden.

\bigskip\noindent
Die Benutzeroberfläche wurde in folgende Abschnitte geteilt:
\begin{itemize}
    \item Login-Seite
    \item Hauptmenü
    \item Optionsmenü
    \item Gruppenmenü
    \item Notizeditor
\end{itemize}

\bigskip\noindent
Die Bedienung der Oberfläche erfolgt mit Maus und Tastatur State Management wird von Pinia, einem Plugin für Vue, übernommen.
Ein Pinia Store muss nur ein mal definiert werden um den State für alle Pages und Komponents, wo der Store angesprochen wird, verfügbar zu machen.

\bigskip\noindent
In Quasar müssen alle Routen in src/router/routes.js angelegt werden, diese werden als JSON-Objekt Array mit den Attributen path, component und children definiert.
Path bezeichnet den URL-Pfad, Component die Vue Komponente, die angezeigt werden soll, und Children können weitere Routen beinhalten die einen Subpfad darstellen.
        
\subsection{Backend}
    Für das Backend wurde das Python-basierte Framework Django benutzt, da der Auszubildende auch hiermit Vorerfahrungen hatte. Django stellt wichtige Funktionen
    wie Nutzeraccounts, Passwort-Hashing und eine Oberfläche zum Verwalten der Datenbank zur Verfügung, welche auch aus Vorerfahrung massive Vorteile für das
    Kontrollieren von API Calls und das Einspielen von Beispieldaten mit sich brachte.

    \bigskip\noindent
    Das Projekt sollte auf einen Live-Server mit einer MariaDB-Datenbank migriert werden, allerdings fehlte nach Entwicklung der Basisfunktionen des Frontends und
    mehreren notwendigen Korrekturen am Datenbankmodell die Zeit hierfür. BrainFlow verwendet die von Django standardmäßig zur Verfügung gestellte SQLite Datenbank.